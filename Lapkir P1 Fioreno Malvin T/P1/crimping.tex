\section{Langkah-Langkah Percobaan}
Bagian ini menjelaskan secara rinci tahapan atau prosedur yang dilakukan selama praktikum. Langkah-langkah ditulis secara urut dan sistematis, mulai dari persiapan alat hingga pelaksanaan percobaan. Penulisan harus jelas agar dapat dipahami oleh orang lain yang membaca laporan ini.

\section{Analisis Hasil Percobaan}
Berisi pembahasan terhadap hasil yang diperoleh selama praktikum. Analisis dilakukan dengan membandingkan hasil percobaan terhadap teori, serta mengidentifikasi faktor-faktor yang memengaruhi hasil, seperti kesalahan alat atau langkah percobaan. Tujuan bagian ini adalah untuk mengevaluasi keberhasilan percobaan dan memastikan pemahaman praktikan terkait praktikum yang telah dilaksanakan.

\section{Hasil Tugas Modul}
Bagian ini memuat hasil dari pengerjaan tugas tambahan yang diberikan dalam modul praktikum serta penjelasan dari jawaban tersebut. 

\section{Kesimpulan}
Kesimpulan berisi ringkasan dari hasil praktikum dan hal-hal penting yang didapatkan. Bagian ini menjawab tujuan praktikum, mencantumkan hasil yang sesuai atau tidak sesuai dengan teori, serta pembelajaran yang diperoleh oleh praktikan.

\section{Lampiran}
\subsection{Dokumentasi saat praktikum}
Menampilkan foto selama pelaksanaan praktikum. Dokumentasi meliputi foto alat yng digunakan dan foto praktikan saat praktikum. Tujuannya sebagai bukti telah dilakukan kegiatan praktikum.

\subsection{Hasil Challenge Modul}
Memuat hasil dari challenge modul saat praktikum (jika berhasil mengerjakan challenge) serta penjelasan lengkap dari hasil tersebut. 
\textcolor{red}{Jika kelompok praktikum tidak berhasil mengerjakan challenge modul, maka sub section ini dapat dihapus.}
