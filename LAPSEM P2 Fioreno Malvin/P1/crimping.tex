\section{Pendahuluan}
\subsection{Latar Belakang}
Dengan banyaknya perangkat yang terhubung ke internet, keterbatasan alamat IPv4 menjadi kendala utama, mendorong pengembangan IPv6 yang kombinasi alamatnya jauh lebih banyak. IPv6 juga harus dirancang agar bisa lebih efisien dan cepat dibandingkan IPv4.

\subsection{Dasar Teori}
Routing dan manajemen IPv6 berkaitan dengan pengiriman data antar perangkat atau jaringan menggunakan IPv6. Teori dasarnya mencakup struktur alamat IPv6 berukuran 128-bit yang mampu menyediakan unique address dalam jumlah yang jauh besar dibandingkan IPV4. Routing IPv6 menggunakan prinsip dasar yang sama dengan IPv4, namun dengan penyederhanaan header yang mempercepat proses forwarding paket oleh perangkat router.

%===========================================================%
\section{Tugas Pendahuluan}
Bagian ini berisi jawaban dari tugas pendahuluan yang telah anda kerjakan, beserta penjelasan dari jawaban tersebut
\begin{enumerate}
	\item IPV6 adalah teknologi terbaru yang dirancang untuk menggantikan IPV4 perlahan lahan, karena jumlah address IPV4 yang sangat terbatas. Panjang alamat IPV4 adalah 32 bit, sedangkan IPV6 128 bit. Notasi alamatnya juga berbeda, IPv4 contohnya 192.168.1.1 dan IPv6 contohnya 2001:0db8:85a3:0000:0000:8a2e:0370:7334. 
	\item A. Subnet 1: 2001:db8:0::/64, Subnet 2: 2001:db8:1::/64, Subnet 3: 2001:db8:2::/64, Subnet 4: 2001:db8:3::/64. B. Subnet 1: 2001:db8:0::1, Subnet 2: 2001:db8:1::1, Subnet 3: 2001:db8:2::1, Subnet 4: 2001:db8:3::1.
	\item ether1: 2001:db8:0::1/64, ether2: 2001:db8:1::1/64, ether3: 2001:db8:2::1/64, ether4: 2001:db8:3::1/64.
        \item Terlampir
        \item Untuk menentukan jalur, meningkatkan keamanan dan memberi ruang overhead. Sebaiknya digunakan di jaringan kecil, atau sebagai backup routing dinamis.

\begin{figure}
    \centering
    \includegraphics[width=0.5\linewidth]{Screenshot 2025-05-16 134028.png}
    \caption{Tabel no. 4}
    \label{fig:enter-label}
\end{figure}

\end{enumerate}
